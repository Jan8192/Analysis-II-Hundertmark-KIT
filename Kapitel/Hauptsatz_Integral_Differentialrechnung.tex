\section{[*] Der Hauptsatz der Integral- und Differentialrechnung}

\subsection{Hauptsatz der Integralrechnung}
\thispagestyle{pagenumberonly}

Sei $I=\interv{a,b}$, $f\in\mC\of{I}$. Wie rechnet man das Integral dann praktisch aus?\\
Erinnerung: $F$ ist eine Stammfunkton von $f$, falls $F$ differenzierbar ist und $F'=f$.

\begin{satz}[Hauptsatz der Differential und Integralrechnung] % Satz 1
    \label{satz:temp-31}
    Sei $f\in\mC\of{I}$. Dann ist für jedes $c\in\interv{a,b}$ die Funktion
    \begin{align*}
        F\of{x} \coloneqq \int_{c}^{x} f\of{t} \dif t\tag{$x\in I$}
    \end{align*}
    stetig differenzierbar und $F' = f$. Das heißt $F'\of{x} = f\of{x}~\fa x\in I$.
    \begin{proof}
    (Später)
    \end{proof}
\end{satz}

\begin{korollar}
    Sei $G\in \mC^{1}\of{I}$ (stetig differenzierbaren Funktionen auf $I$) eine Stammfunktion von $f\in\mC\of{I}$. Dann gilt
    \begin{align*}
        \int_{a}^{b} f\of{x} \dif x = G\of{b} - G\of{a} \eqqcolon G\vert_a^b \coloneqq \interv{G}_a^b = \interv{G\of{x}}_{x=a}^{x=b}
    \end{align*}
    \begin{proof}
        Wir nehmen $c= a$ aus Satz~\ref{satz:temp-31} und $F: I\fromto\R~x\mapsto F\of{x} = \int_{a}^{x} f\of{t} \dif t$ erfüllt $F' = f$ auf $I$ nach Satz~\ref{satz:temp-31}.
        \begin{align*}
            F\of{b} &= \int_{a}^{b} f\of{t} \dif t\\
            h\of{t} &\coloneqq F\of{t} - G\of{t}\\
            h' &= F' - G' = f-f = 0 \text{ auf } I
            \intertext{Damit ist $h$ konstant, d.h. $h\of{x} = k$ für alle $x\in I$}
            \impl F\of{x} - G\of{x} &= k\\
            k &= F\of{a} - G\of{a} = -G\of{a}\\
            F\of{x} - G\of{x} &= -G\of{a}\\
            F\of{x} &= G\of{x} - G\of{a}\\
            \impl F\of{b} &= G\of{b} - G\of{a}\qedhere
        \end{align*}
    \end{proof}
\end{korollar}

\begin{proof}[Beweis von Satz~\ref{satz:temp-31}]
    \marginnote{[3. Mai]}
    Sei $F(x) = \int_{c}^{x} f\of{t} \dif t$ und $h \neq 0$. Wir wollen über den Differenzenquotient zeigen, dass $F' = f$. Wir berechnen zuerst den Zähler
    \begin{align*}
        F(x+h) - F(x) &= \int_{c}^{x+h} f\of{t} \dif t - \int_{c}^{x} f\of{t} \dif t = \int_{x}^{x+h} f\of{t} \dif t
        \intertext{Das können wir in den Differenzenquotienten einsetzen}
        \frac{F\of{x+h} - F\of{x}}{h} &= \frac{1}{h}\cdot \int_{x}^{x+h} f\of{t} \dif t\\
        \impl \frac{F\of{x+h} - F\of{x}}{h} - f\of{x} &= \frac{1}{h}\cdot \int_{x}^{x+h} f\of{t} \dif t - f\of{x}\\
        &= \frac{1}{h} \int_{x}^{x+h} f\of{t} \dif t - \frac{1}{h} \int_{x}^{x+h} f\of{x} \dif t\\
        &= \frac{1}{h} \int_{x}^{x+h} \pair{f\of{t} - f\of{x}} \dif t\\
        \impl \abs{\frac{F\of{x+h} - F\of{x}}{h} - f\of{x}} &\leq \begin{cases}
                                                                      \frac{1}{h} \int_{x}^{x+h} \pair{f\of{t} - f\of{x}} \dif t &h>0\\
                                                                      \frac{1}{-h} \int_{x}^{x+h} \pair{f\of{t} - f\of{x}} \dif t &h<0
        \end{cases}\\
        &\leq \frac{1}{\abs{h}} \cdot \sup_{x \leq t \leq x +h} \abs{f\of{t} - f\of{x}} \cdot \abs{h}
        \intertext{Wir definieren $I_h\of{x} = \interv{x, x+h}$, falls $h> 0$ und ansonsten $I_h\of{x} = \interv{x+h, x}$}
        &\leq \sup_{t\in I_h\of{x}} \abs{f\of{t} - f\of{x}}
        \intertext{Da $f$ stetig in $x$ ist, folgt}
        \sup_{t\in I_h\of{x}} \abs{f\of{x} - f\of{x}} &\fromto 0 \text{ für } h\fromto 0\\
        \impl \lim_{h\fromto 0} \abs{\frac{F\of{x+h} - F\of{x}}{h} - f\of{x}} &= 0\\
        \equivalent \lim_{h\fromto 0} \frac{F\of{x+h} - F\of{x}}{h} &= f\of{x}\qedhere
    \end{align*}
\end{proof}

\begin{beispiel}
    Sei $p\in\N$ und $f\of{x} = x^{p}$, $x\in\R$. Dann hat $f$ die Stammfunktion $F\of{x} = \frac{1}{p+1}\cdot x^{p+1}$. Damit folgt
    \begin{align*}
        \int_{a}^{b} x^p \dif x &= \frac{1}{p+1} \cdot \interv{b^{p+1} - a^{p+1}}\quad\forall a, b\in\R
    \end{align*}
\end{beispiel}
\begin{beispiel}
    Sei $p\in\N$, $p\geq 2$ und $f\of{x} = x^{-p}$, $x\neq 0$. Dann ist die Stammfunktion $F\of{x} = \frac{1}{1-p}\cdot x^{1-p}$. Damit folgt
    \begin{align*}
        \int_{a}^{b} x^{-p} \dif x &= \frac{1}{1-p}\cdot\interv{b^{1-p} - a^{1-p}}\quad \fa a,b < 0 \text{ oder } a,b > 0
    \end{align*}
\end{beispiel}
\begin{beispiel}
    Sei $\alpha\in\R\exclude\set{-1}$, $f\of{x} = x^{\alpha} = e^{\alpha\cdot\ln\of{x}}$, $x > 0$. Dann ist die Stammfunktion $F\of{x} = \frac{1}{\alpha + 1}\cdot x^{\alpha + 1}$. Damit gilt
    \begin{align*}
        \int_{a}^{b} x^{\alpha} \dif x &= \frac{1}{\alpha +1}\cdot\interv{b^{\alpha+1} - a^{\alpha+1}}\quad\forall a,b > 0
    \end{align*}
\end{beispiel}
\begin{beispiel}
    Sei $f\of{x} = \frac{1}{x}$, $x\neq 0$. Dann ist die Stammfunktion $F\of{x} = \ln \abs{x}$.
    \begin{proof}
        Falls $x > 0$. Dann ist $F\of{x} = \ln x$ und $F'\of{x} = \frac{1}{x}$.\\
        Falls $x < 0$. Dann ist $F\of{x} = \ln -x$ und $F'\of{x} = \frac{1}{-x}\cdot\pair{-1} = \frac{1}{x}$.
    \end{proof}
    \noindent Damit gilt
    \begin{align*}
        \int_{a}^{b} \frac{1}{x} \dif x &= \ln\abs{b} - \ln\abs{a} = \ln\abs{\frac{b}{a}}\quad\forall a,b < 0 \text{ oder } a,b > 0
    \end{align*}
\end{beispiel}

\begin{beispiel}
    Es gilt $\pair{\sin x}' = \cos x$ und $\pair{\cos x}' = -\sin x$. Damit gilt
    \begin{align*}
        \int_{a}^{b} \cos x \dif x &= \sin b - \sin a\\
        \int_{a}^{b} \sin x \dif x &= \interv{-\cos x}_a^b -\cos b + \cos a\\
    \end{align*}
\end{beispiel}

\begin{beispiel}
    Es gilt $\tan x = \frac{\sin x}{\cos x}$. ($\abs{x} < \frac{\pi}{2}$). Damit folgt $\pair{\tan x}' = \frac{1}{\cos^2 x}$. Das heißt
    \begin{align*}
        \int_{0}^{\varphi} \frac{1}{\cos^2 x} \dif x &= \interv{\tan x}_0^{\varphi} = \tan\of{\varphi}\quad\forall\abs{\varphi} < \frac{\pi}{2}
    \end{align*}
\end{beispiel}

\begin{beispiel}
    Wir wollen das Integral $ \int_{a}^{b} \sqrt{1-x^2} \dif x$ berechnen. $\sqrt{1-x^2}$ hat die Stammfunktion $\phi\of{x} = \frac{1}{2}\pair{\arcsin x + x\cdot\sqrt{1-x^2}}$, weil
    \begin{align*}
        \phi'\of{x} &= \frac{1}{2}\pair{\frac{1}{\sqrt{1-x^2}} + \sqrt{1-x^2} + x\cdot \frac{1}{2\sqrt{1-x^2}}\pair{-2x}}\tag{$\pair{\arcsin\of{x}}' = \frac{1}{\sqrt{1-x^2}}$}\\
        &= \frac{1}{2}\pair{\frac{1}{\sqrt{1-x^2}} + \sqrt{1-x^2} - \frac{x^2}{\sqrt{1-x^2}}}\\
        &= \frac{1}{2}\pair{\frac{1-x^2}{\sqrt{1-x^2}} + \sqrt{1-x^2}} = \frac{1}{2}\pair{\sqrt{1-x^2} + \sqrt{1-x^2}} = \sqrt{1-x^2}\\
        \impl \int_{a}^{b} \sqrt{1-x^2} \dif x &= \interv{\frac{1}{2}\pair{\arcsin x + x\cdot\sqrt{1-x^2}}}_a^b\quad -1\leq a,b\leq 1
        \intertext{Geometrisch gesehen können wir damit auch die Fläche der oberen Hälfte des Einheitskreises berechnen}
        \int_{-1}^{1} \sqrt{1-x^2} \dif x &= \frac{1}{2} \cdot\pair{\arcsin 1 + 0 - \arcsin -1 -0} = \arcsin 1 = \frac{\pi}{2}
    \end{align*}
\end{beispiel}

\begin{bemerkung}
    Satz~\ref{satz:temp-31} gilt auch für Funktionen in $\C$ oder $\R^d$ bwz. $\C^d$. Wir nennen
    \begin{align*}
        \int_{}^{} f\of{x} \dif x
        \intertext{die Gesamtheit aller Stammfunktionen zu $f$ oder das unbestimmte Integral. Genauer gilt, wenn $\Phi$ eine Stammfunktion von $f$ ist}
        \int_{}^{} f\of{x} \dif x &= \set{\Phi + k: k \text{ Konstante} }
    \end{align*}

\end{bemerkung}

\subsection{Integrationstechniken}
\begin{satz}[Partielle Integration] % Satz 3
    \label{satz:temp-33}
    Seien $f, g\in \mC^1\of{I}$ (oder $\mC^1\of{I, \C}$). Dann gilt
    \begin{align*}
        \int_{a}^{b} f'\of{x}g\of{x} \dif x &= \interv{f\of{x}\cdot g\of{x}}_a^b - \int_{a}^{b} f\of{x}g'\of{x} \dif x\\
        &= f\of{b}g\of{b} - f\of{a}g\of{a} - \int_{a}^{b} f\of{x}g'\of{x} \dif x
    \end{align*}
    \begin{proof}
        Wir wenden die Produktregel der Ableitung an. Es gilt $\pair{fg}' = f'g + fg'$.
        \begin{align*}
            \int_{a}^{b} \pair{fg}' \dif x &= \int_{a}^{b} f'g \dif x + \int_{a}^{b} fg' \dif x\tag{1}
            \intertext{Außerdem gilt}
            \int_{a}^{b} \pair{fg}' \dif x &= \interv{fg}_a^b = \interv{f\of{x}g\of{x}}_{a}^b = f\of{b}g\of{b} - f\of{a}g\of{a}\tag{2}
            \intertext{Wir setzen (1) und (2) gleich}
            \int_{a}^{b} f'g \dif x + \int_{a}^{b} fg' \dif x &= f\of{b}g\of{b} - f\of{a}g\of{a}\qedhere
        \end{align*}
    \end{proof}
\end{satz}

\begin{beispiel}[Anwendung von partieller Integration]
    \begin{align*}
        \int_{}^{} \ln x \dif x &= \int_{}^{} 1\cdot\ln x \dif x = x\cdot\ln x - \int_{}^{} x\cdot \frac{1}{x} \dif x\\
        &= x\cdot\ln x - x\\[5pt]
        \int_{}^{} \sqrt{1-x^2} \dif x &= \int_{}^{} 1\cdot\sqrt{1-x^2} \dif x\\
        &= x\cdot\sqrt{1-x^2} - \int_{}^{} x\cdot\frac{1}{\sqrt{1-x^2}}\cdot\pair{-2x} \dif x\\
        &= x\cdot\sqrt{1-x^2} + \int_{}^{} \frac{x^2}{\sqrt{1-x^2}} \dif x\\
        &= x\cdot\sqrt{1-x^2} + \int_{}^{} \frac{1}{\sqrt{1-x^2}} \dif x - \int_{}^{} \frac{1-x^2}{\sqrt{1-x^2}} \dif x\\
        \impl 2 \int_{}^{} \sqrt{1-x^2} \dif x &= x\cdot\sqrt{1-x^2} + \int_{}^{} \frac{1}{\sqrt{1-x^2}} \dif x\\
        &= x\cdot\sqrt{1-x^2} + \arcsin x\\
        \impl \int_{}^{} \sqrt{1-x^2} \dif x &= \frac{1}{2}\pair{x\sqrt{1-x^2} + \arcsin x}
    \end{align*}
\end{beispiel}

\begin{uebung}
    Beweisen Sie analog zum vorherigen Beispiel mittels partieller Integration, dass $ \int_{}^{} \sqrt{1+x^2} \dif x = \frac{1}{2}\pair{x\sqrt{1+x^2} + \arcsinh x}$ und $ \int_{}^{} \sqrt{x^2-1} \dif x = \frac{1}{2}\pair{x\sqrt{1+x^2} + \arccosh x}$
\end{uebung}
\begin{beispiel}
    \begin{align*}
        \int_{}^{} e^{ax}\cdot \sin\of{bx} \dif x &= e^{ax}\cdot\pair{-\frac{1}{b}\cos\of{bx}} - \int_{}^{} \frac{a}{b} e^{ax} \cos\of{bx} \dif x
        \intertext{Wir wenden nochmal partielle Integration an und erahlten}
        &= -\frac{1}{b}e^{ax} \cos\of{bx} + \frac{a}{b}\set{\int_{}^{} e^{4x} \cos\of{bx} \dif x}\\
        &= -\frac{1}{b}e^{ax} \cos\of{bx} + \frac{a}{b}\set{\frac{1}{b} e^{ax} \sin\of{bx} - \frac{a}{b}\int_{}^{} e^{ax}\sin\of{bx} \dif x}\\
        \impl \pair{1+\frac{a^2}{b^2}} \int_{}^{} e^{ax}\sin\of{bx} \dif x &= -\frac{1}{b} e^{ax}\cos\of{bx} + \frac{a}{b}e^{ax}\sin\of{bx}\\
        \impl \int_{}^{} e^{ax}\sin\of{bx} \dif x &= \frac{1}{a^2+b^2}\pair{e^{ax}\pair{a\sin\of{bx} - b\cos\of{bx}}} + const.
    \end{align*}
\end{beispiel}

\begin{beispiel}
    \marginnote{[07. Mai]}
    \begin{align*}
        \int_{0}^{\frac{\pi}{2}} \cos^2\of{x} \dif x &= \int_{0}^{\frac{\pi}{2}} \sin^2\of{x} \dif x = \frac{\pi}{4}\\
        \int_{0}^{\frac{\pi}{2}} \sin^2\of{x} \dif x &= \int_{0}^{\frac{\pi}{2}} \sin\of{x}\sin\of{x} \dif x\\
        &= \interv{-\cos\of{x}\sin\of{x}}_{0}^{\frac{\pi}{2}} + \int_{0}^{\frac{\pi}{2}} \cos\of{x}\cos\of{x} \dif x\\
        &= 0 - 0 + \int_{0}^{\frac{\pi}{2}} \cos^2\of{x} \dif x
        \intertext{Mit dem trigonometrischen Pythagoras wissen wir außerdem, dass}
        \frac{\pi}{2} &= \int_{0}^{\frac{\pi}{2}}  \dif x = \int_{0}^{\frac{\pi}{2}} 1 \dif x = \int_{0}^{\frac{\pi}{2}} \pair{\cos^2\of{x} + \sin^2\of{x}} \dif x\\
        &= \int_{0}^{\frac{\pi}{2}} \cos^2\of{x} \dif x + \int_{0}^{\frac{\pi}{2}} \sin^2\of{x} \dif x\\
        &= 2 \int_{0}^{\frac{\pi}{2}} \cos^2\of{x} \dif x\\
        \impl \int_{0}^{\frac{\pi}{2}} \cos^2\of{x} \dif x &= \frac{\pi}{4}
    \end{align*}
\end{beispiel}

\begin{beispiel}
    Sei $n\in\N$ mit $n\geq 2$
    \begin{align*}
        \int_{}^{} \cos^{n}\of{x} \dif x &= \int_{}^{} \cos\of{x}\cos^{n-1}\of{x} \dif x\\
        &= \sin\of{x}\cos^{n-1}\of{x} + \int_{}^{} \sin\of{x}\pair{n-1}\cos^{n-2}\of{x}\sin\of{x} \dif x\\
        &= \sin\of{x}\cos^{n-1}\of{x} + \pair{n-1} \int_{}^{} \underbrace{\sin^2\of{x}}_{= 1- \cos^2\of{x}}\cos^{n-2}x \dif x\\
        &= \sin\of{x}\cos^{n-1}\of{x} + \pair{n-1} \int_{}^{} \cos^{n-2}x \dif x - \pair{n-1} \int_{}^{} \cos^{n}\of{x}\dif x\\
        \int_{}^{} \cos^{n}x \dif x &= \frac{1}{n} \sin\of{x}\cos^{n-1}\of{x} + \frac{n-1}{n} \int_{}^{} \cos^{n-2}\of{x}\dif x\tag{Rekursionsformel}
        \intertext{Analog lässt sich zeigen, dass $ \int_{}^{} \sin^{n}\of{x}x \dif x = \frac{1}{n}\cos\of{x}\sin^{n-1}\of{x} + \frac{n-1}{n} \int_{}^{} \sin^{n-2}\of{x} \dif x$. Wir nutzen nun die Rekursionsformel, um einen Wert für alle $n$ zu ermitteln}
        c_n &\coloneqq \int_{0}^{\frac{\pi}{2}} \cos^{n}\of{x} \dif x\\
        &= \interv{\frac{1}{n}\sin\of{x}\cos^{n-1}\of{x}}_0^{\frac{\pi}{2}} + \frac{n-1}{n} \int_{0}^{\frac{\pi}{2}} \cos^{n-2}\of{x} \dif x\\
        &= \frac{n-1}{n} \underbrace{\int_{0}^{\frac{\pi}{2}} \cos^{n-2}\of{x} \dif x}_{= c_{n-2}}\\
        \impl c_n &= \frac{n-1}{n} c_{n-2}\quad\forall n\geq 2\\
        c_0 &= \frac{\pi}{2}\\
        c_1 &= \int_{0}^{\frac{\pi}{2}} \cos\of{x} \dif x = \interv{\sin\of{x}}_0^{\frac{\pi}{2}} = 1-0 = 1\\
        c_n &= \frac{n-1}{n}\cdot c_{n-2} = \frac{n-1}{n}\cdot\frac{n-3}{n-2}\cdot c_{n-4}\\
        &= \frac{n-1}{n}\cdot\ldots\cdot\frac{n-j-1}{n-j}\cdot c_{n-2j-2}\qquad\forall j: n-2j-2 \geq 1
        \intertext{Damit folgt für $k\in\N$}
        c_{2k} &= \frac{2k-1}{2k} \cdot\frac{2k-3}{2k-2}\cdot \ldots\cdot\frac{3}{4}\cdot \int_{0}^{\frac{\pi}{2}} \cos^2\of{x} \dif x\\
        &= \frac{2k-1}{2k} \cdot\frac{2k-3}{2k-2}\cdot \ldots\cdot\frac{3}{4}\cdot\frac{1}{2}\cdot\frac{\pi}{2}\\
        c_{2k+1} &= \frac{2k}{2k+1}\cdot\frac{2k-3}{2k-2}\cdot\ldots\cdot\frac{2\cdot 2}{5}\cdot \int_{0}^{\frac{\pi}{2}} \cos^3\of{x} \dif x\\
        &= \frac{2k}{2k+1}\cdot\frac{2k-3}{2k-2}\cdot\ldots\cdot\frac{2\cdot 2}{5}\cdot \frac{2}{3}
    \end{align*}
\end{beispiel}

\begin{satz}[Wallisches Produkt] % Satz 3
    Sei $n\in\N$ und
    \begin{align*}
        W_n &\coloneqq \frac{2\cdot 2}{1\cdot 3}\cdot \frac{4\cdot 4}{3\cdot 5}\cdot \ldots\cdot\frac{2n\cdot 2n}{\pair{2n-1}\cdot\pair{2n+1}}
        \intertext{Dann gilt}
        \lim_{\ntoinf} W_n &= \frac{\pi}{2}
    \end{align*}
    \begin{proof}
        Aus der Definition von $c_n$ aus dem vorherigen Beispiel ergibt sich
        \begin{align*}
            W_n &= \frac{\pi}{2} \cdot \frac{c_{2n+1}}{c_{2n}}
            \intertext{Für $x\in\interv{0, \frac{\pi}{2}}$ ist $0\leq\cos\of{x}\leq 1$. Damit folgt $\cos^{2n}\of{x} \leq \cos^{2n-1}\of{x}\leq \cos^{2n-}2\of{x}$. Also gilt}
            c_{2n} &= \int_{0}^{\frac{\pi}{2}} \cos^{2n}\of{x} \dif x \leq \int_{0}^{\frac{\pi}{2}} \cos^{2n-1}\of{x} \dif x \leq \int_{0}^{\frac{\pi}{2}} \cos^{2n-2} \dif x\\
            \impl c_{2n} &\leq c_{2n-1} \leq c_{2n-2}\qquad\forall n\in\N
            \intertext{Nach Def. gilt}
            c_{2n} &= \frac{\pi}{2}\cdot\prod_{j=1}^{k} \frac{2j-1}{2j}\\
            \impl \frac{c_{2n+2}}{c_{2n}} &= \frac{\frac{\pi}{2}\cdot \prod_{j=1}^{n+1} \frac{2j-1}{2j}}{\frac{\pi}{2}\cdot \prod_{j=1}^{n} \frac{2j-1}{2j}} = \frac{2\pair{n+1}-1}{2\pair{n+1}} = \frac{2n+1}{2n+2}\fromto 1 \text{ für } \ntoinf
            \intertext{Auch}
            1 &= \frac{c_{2n}}{c_{2n}} \geq \abs{\frac{c_{2n+1}}{c_2n}} \geq \frac{c_{2n+2}}{c_{2n}} = \frac{2n+1}{2n+2}\\
            &\impl \lim_{\ntoinf} \frac{c_{2n+1}}{c_n} = 1\qedhere
        \end{align*}
    \end{proof}
    Außerdem
    \begin{align*}
        W_n &= \frac{2^2\cdot 4^2\cdot 6^2\cdot\ldots\cdot \pair{2n-2}^2}{3^2\cdot 5^2\cdot 7^2\cdot\dots\cdot\pair{2n-1}^2} \cdot 2n \cdot \frac{2n}{2n+1}\\
        \impl \sqrt{W_n} &= \frac{2\cdot 4 \cdot \ldots \pair{2n-2}}{3\cdot 5 \cdot\ldots\cdot \pair{2n-1}}\cdot\sqrt{2n}\cdot\sqrt{\frac{2n}{2n+1}}\\
        \impl \sqrt{\frac{\pi}{2}} &= \lim_{\ntoinf} \frac{2\cdot 4\cdot\ldots\cdot\pair{2n-2}}{3\cdot 5 \cdot\ldots \cdot\pair{2n-1}}\cdot\sqrt{2n}\\
        &= \lim_{\ntoinf} \frac{2^2 \cdot 4^2\cdot \ldots \cdot \pair{2n-2}^2}{2\cdot 3 \cdot \ldots \cdot \pair{2n-2}\cdot\pair{2n-1}}\cdot\sqrt{2n}\\
        &= \frac{2^2\cdot 4^2\cdot\ldots \cdot \pair{2n-2}^2\cdot\pair{2n}^2}{\pair{2n-1}!\cdot 2n\cdot\sqrt{2n}}\\
        &= \frac{2^{2n}\cdot\pair{n!}^2}{\pair{2n}!\cdot\sqrt{2n}} = \frac{2^{2n}}{\binom{2n}{n}\sqrt{n}} \frac{1}{\sqrt{2}}\\
        \impl \sqrt{\pi} &= \lim_{\ntoinf} \frac{2^{2n}}{\binom{2n}{n}\sqrt{n}}
    \end{align*}
\end{satz}

\begin{satz}[Substitutionsregel] % Satz 4
    Seien $I=\interv{a,b}$ und $I^{*}$ kompakte Intervalle und $f\in\mC\of{I, \C}$, $\varphi\in\mC^{1}\of{I^{*}, \R}$ sowie $\varphi\of{I^{*}}\subseteq I$. Dann gilt für $\alpha, \beta \in I^{*}$
    \begin{align*}
        \int_{\varphi\of{\alpha}}^{\varphi\of{\beta}} f\of{x} \dif x &= \int_{\alpha}^{\beta} f\of{\varphi\of{t}}\cdot\varphi'\of{t} \dif t
    \end{align*}
    \begin{proof}
        Sei $F$ die Stammfunktion von $f$ ($F'\of{x} = f\of{x}~\forall x\in I$). Wir definieren $h\of{t}\coloneqq F\of{\varphi\of{t}} \impl h\in\mC^{1}\of{I^{*}, \C}$ (Kettenregel).
        \begin{align*}
            h'\of{t} &= \frac{\dif}{\dif t} h\of{t} = F'\of{\varphi\of{t}}\cdot\varphi'\of{t} = f\of{\varphi\of{t}}\cdot\varphi'\of{t}\\
            \int_{\alpha}^{\beta} h'\of{t} \dif t &= \interv{h\of{t}}_{\alpha}^{\beta} = h\of{\beta}-h\of{\alpha} = F\of{\varphi\of{\beta}} - F\of{\varphi\of{\alpha}}\\
            &= \int_{\varphi\of{\alpha}}^{\varphi\of{\beta}} F'\of{x} \dif x = \int_{\varphi\of{\alpha}}^{\varphi\of{\beta}} f\of{x} \dif x\qedhere
        \end{align*}
    \end{proof}
    \noindent \textsc{Erste Lesart:} $ \int_{\alpha}^{\beta} g\of{t} \dif t$ ausrechnen. Annahme: Es existiert eine Substitution $x=\varphi\of{t}$ und $f\of{x}$, sodass $g\of{t} = f\of{\varphi\of{t}}\cdot\varphi'\of{t}$ ist.
    \begin{align*}
        \impl \int_{\alpha}^{\beta} g\of{t} \dif t &= \int_{a}^{b} f\of{x} \dif x\tag{$b=\varphi\of{\beta}$, $a = \varphi\of{\alpha}$}
    \end{align*}
\end{satz}

\begin{beispiel}
    Wir betrachten des Integral $\int_{\alpha}^{\beta} g\of{t+c} \dif t$. Wir definieren $\varphi\of{t} = t+c$ und $f\of{x} = g\of{x}$. Dann gilt $\varphi'\of{t} = 1$
    \begin{align*}
        \impl \int_{\alpha}^{\beta} g\of{t+c} \dif t &= \int_{\alpha}^{\beta} g\of{\varphi\of{t}}\cdot\varphi'\of{t} \dif t\\
        &= \int_{\varphi\of{\alpha}}^{\varphi\of{\beta}} g\of{x} \dif x\\
        &= \int_{a+c}^{b+c} g\of{x} \dif x\tag{Translation}
    \end{align*}
\end{beispiel}

\begin{beispiel}
    Wir betrachten $\int_{a}^{b} g\of{t} \frac{\dif t}{t}$ mit $a,b > 0$ und definieren $\varphi\of{t} = \ln\of{t}$, $\varphi'\of{t} = \frac{1}{t}$.
    \begin{align*}
        g\of{t}\cdot \frac{1}{t} &= g\of{t}\cdot\varphi'\of{t}\\
        &= g\of{e^{\varphi\of{t}}}\cdot\varphi'\of{t}\\
        f'\of{x} &= g\of{e^x}\\
        \impl \int_{a}^{b} g\of{t} \frac{\dif t}{t} &= \int_{a}^{b} f\of{\varphi\of{t}}\cdot\varphi'\of{t} \dif t\\
        &= \int_{\varphi\of{\alpha}}^{\varphi\of{beta}} f\of{x} \dif x = \int_{\ln a}^{\ln b} f\of{x} \dif x\\
        &= \int_{\ln a}^{\ln b} g\of{e^x} \dif x
    \end{align*}
\end{beispiel}

 \begin{beispiel}
     Wir betrachten $ \int_{0}^{1} \pair{1+t^2}^{n}\cdot (t \dif x$. $t = \frac{1}{2}\frac{\dif}{\dif t}\pair{1+t^2}$. $\pair{1+t^2}^n = \frac{1}{2}\pair{1+t^2}^n \frac{\dif}{\dif t}\pair{1+t^2}$. $\varphi\of{t} = 1+t^2$, $f\of{x} = \frac{1}{2}x^n$.\\
     Dann gilt $\pair{1+t^2}^n = \frac{1}{2}f\of{\varphi\of{t}}\cdot\varphi'\of{t}$.
 \end{beispiel}

\begin{genv}
    \marginnote{[10. Mai]}
    Ziel: Berechne $ \int_{a}^{b} f\of{x} \dif x$. Wir führen eine Variablentransformation durch: $x=\varphi\of{t}$, $\alpha \leq t \leq \beta$.
    \begin{align*}
        \leadsto \int_{\alpha}^{\beta} h\of{t} \dif x\tag{$h\of{t} = f\of{\varphi\of{t}}\cdot\varphi'\of{t}$}\\
        \impl \int_{a}^{b} f\of{x} \dif x &= \int_{\varphi^{-1}\of{a}}^{\varphi^{-1}\of{b}} h\of{t} \dif t\\
        &= \int_{\varphi^{-1}\of{a}}^{\varphi^{-1}\of{b}} f\of{\varphi\of{t}}\cdot\varphi'\of{t} \dif t
    \end{align*}
    Dazu benötigt man $\varphi: \interv{\alpha, \beta}\fromto\interv{a,b}$ ist invertierbar. (Also zum Beispiel $\varphi' > 0$ oder $\varphi' < 0$ auf ganz $\interv{\alpha, \beta}$)
\end{genv}

\begin{notation}[Leibnitz'sche Schreibweise]
    $x=\varphi\of{t}$\quad $\frac{\dif x}{\dif t} = \varphi'\of{t}$ (informell).\\
    $\dif x$ \anf{=} $\varphi'\of{t}\dif t$
    \begin{align*}
        \impl \int_{}^{} f\of{x} \dif x &\text{\anf{=}} \int_{}^{} f\of{\varphi\of{t}}\cdot\varphi'\of{t} \dif t\\
        \int_{0}^{1} \sqrt{r^2-x^2} \dif x &= \int_{0}^{\frac{\pi}{2}} \sqrt{r^2-r^2\sin^2\of{t}}\cdot\cos\of{t} \dif t\tag{$\frac{\dif x}{\dif t} = r\cdot\cos t$}\\
        &= r^2\cdot\frac{\pi}{4}
    \end{align*}
\end{notation}

\newpage
