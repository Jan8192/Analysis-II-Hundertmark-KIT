\documentclass[11pt, twoside, a4paper]{article}

% Setup
\usepackage[margin=2.4cm, top=3.5cm]{geometry}
\usepackage[utf8]{inputenc}
\usepackage[ngerman]{babel}

% Package imports
\usepackage{amsfonts}
\usepackage{amsmath}
\usepackage{amssymb}
\usepackage{amsthm}
\usepackage{mathtools}
\usepackage{setspace}
\usepackage{float}
\usepackage{enumitem}
\usepackage{hyperref}
\usepackage[pagestyles]{titlesec}
\usepackage{fancyhdr}
\usepackage{colonequals}
\usepackage{caption}
\usepackage{tikz}
\usepackage{marginnote}
\usepackage{etoolbox}
\usepackage{mdframed}
\usepackage{aligned-overset}

% Font-Encoding
\usepackage[T1]{fontenc}
\usepackage{lmodern}
\usepackage{watermark}

% Theorems
\newtheorem{blockelement}{Blockelement}[subsection]
\newtheoremstyle{plain}{}{}{}{}{\bfseries}{.}{ }{}
\theoremstyle{plain}
\newtheorem{bemerkung}[blockelement]{Bemerkung}
\newtheorem{definition}[blockelement]{Definition}
\newtheorem{lemma}[blockelement]{Lemma}
\newtheorem{satz}[blockelement]{Satz}
\newtheorem{notation}[blockelement]{Notation}
\newtheorem{korollar}[blockelement]{Korollar}
\newtheorem{uebung}[blockelement]{Übung}
\newtheorem{beispiel}[blockelement]{Beispiel}
\newtheorem{folgerung}[blockelement]{Folgerung}
\newtheorem{axiom}[blockelement]{Axiom}
\newtheorem{beobachtung}[blockelement]{Beobachtung}
\newtheorem{konzept}[blockelement]{Konzept}
\newtheorem{visualisierung}[blockelement]{Visualisierung}
\newtheorem{anwendung}[blockelement]{Anwendung}
\newtheorem{skizze}[blockelement]{Skizze}
\newtheorem{genv}[blockelement]{}

% Marginnotes left
\makeatletter
\patchcmd{\@mn@@@marginnote}{\begingroup}{\begingroup\@twosidefalse}{}{\fail}
\reversemarginpar
\makeatother

% Long equations
\allowdisplaybreaks

% \left \right
\newcommand{\set}[1]{\left\{#1\right\}}
\newcommand{\pair}[1]{\left(#1\right)}
\newcommand{\of}[1]{\mathopen{}\mathclose{}\bgroup\left(#1\aftergroup\egroup\right)}
\newcommand{\abs}[1]{\left\lvert#1\right\rvert}
\newcommand{\norm}[1]{\left\lVert#1\right\rVert}
\newcommand{\linterv}[1]{\left[#1\right)}
\newcommand{\rinterv}[1]{\left(#1\right]}
\newcommand{\interv}[1]{\left[#1\right]}
\newcommand{\sprod}[1]{\left<#1\right>}

% Shorten commands
\newcommand{\equivalent}[0]{\Leftrightarrow{}}
\newcommand{\impl}[0]{\Rightarrow{}}
\newcommand{\fromto}{\rightarrow{}}
\newcommand{\definedas}[0]{\coloneqq}
\newcommand{\definedasbackwards}[0]{\eqqcolon}
\newcommand{\definedasequiv}[0]{\ratio\Leftrightarrow{}}
\newcommand{\exclude}[0]{\setminus}
\renewcommand{\emptyset}{\varnothing}
\newcommand{\sbset}{\subseteq}
\newcommand{\dif}{\mathop{}\!\mathrm{d}}

\newcommand{\ntoinf}[0]{n\fromto\infty}
\newcommand{\toinf}{\fromto\infty}
\newcommand{\fa}{\;\forall\,}
\newcommand{\ex}{\;\exists\,}
\newcommand{\conj}[1]{\overline{#1}}

\newcommand{\annot}[3][]{\overset{\text{#3}}#1{#2}}
\newcommand{\biglim}[1]{{\displaystyle \lim_{#1}}}
\newcommand{\nn}[0]{\\[2\baselineskip]}
\newcommand{\anf}[1]{\glqq{}#1\grqq}
\newcommand{\OBDA}{o.B.d.A. }
\newcommand{\theoremescape}{\leavevmode}
\newcommand{\aligntoright}[2]{\hfill#1\hspace{#2\textwidth}~}
\newcommand{\horizontalline}[0]{\par\noindent\rule{0.05\textwidth}{0.1pt}\\}
\newcommand{\rgbcolor}[3]{rgb,255:red,#1;green,#2;blue,#3}
\newcommand{\fixedspace}[2]{\makebox[#1][l]{#2}}

\let\Re\relax
\let\Im\relax

% MathOperators
\DeclareMathOperator{\grad}{Grad}
\DeclareMathOperator{\bild}{Bild}
\DeclareMathOperator{\Re}{Re}
\DeclareMathOperator{\Im}{Im}

% Mengenbezeichner
\newcommand{\R}{\mathbb{R}}
\newcommand{\N}{\mathbb{N}}
\newcommand{\C}{\mathbb{C}}
\newcommand{\Z}{\mathbb{Z}}
\newcommand{\Q}{\mathbb{Q}}
\newcommand{\K}{\mathbb{K}}

\newcommand\imaginarysubsection[1]{
    \refstepcounter{subsection}
    \subsectionmark{#1}
}

% Unfassbar hässlich, aber effektiv für temporäre schnelle Lösungen
\def\:={\coloneqq}
\def\->{\fromto}
\def\=>{\impl}
\def\<={\leq}
\def\>={\geq}

% Envs
\newenvironment{induktionsanfang}{
    \rule{0pt}{3ex}\noindent
    \begin{minipage}[t]{0.11\textwidth}
    {I-Anfang}
    \end{minipage}
    \hfill
    \begin{minipage}[t]{0.89\textwidth}
    }
    {
    \end{minipage}
}
\newenvironment{induktionsvoraussetzung}{
    \rule{0pt}{3ex}\noindent
    \begin{minipage}[t]{0.11\textwidth}
    {I-Vor.}
    \end{minipage}
    \hfill
    \begin{minipage}[t]{0.89\textwidth}
    }
    {
    \end{minipage}
}
\newenvironment{induktionsschritt}{
    \rule{0pt}{3ex}\noindent
    \begin{minipage}[t]{0.11\textwidth}
    {I-Schritt}
    \end{minipage}
    \hfill
    \begin{minipage}[t]{0.89\textwidth}
    }
    {
    \end{minipage}
}

% Section style
\titleformat*{\section}{\LARGE\bfseries}
\titleformat*{\subsection}{\large\bfseries}

% Page styles
\newpagestyle{pagenumberonly}{
    \sethead{}{}{}
    \setfoot[][][\thepage]{\thepage}{}{}
}
\newpagestyle{headfootdefault}{
    \sethead[][][\thesubsection~\textit{\subsectiontitle}]{\thesection~\textit{\sectiontitle}}{}{}
    \setfoot[][][\thepage]{\thepage}{}{}
}
\pagestyle{headfootdefault}

\begin{document}
    \title{\vspace{3cm} Skript zur Vorlesung\\Analysis II\\bei Prof. Dr. Dirk Hundertmark}
    \author{Karlsruher Institut für Technologie}
    \date{Sommersemester 2024}
    \maketitle
    \begin{center}
        Dieses Skript ist inoffiziell. Es besteht kein\\Anspruch auf Vollständigkeit oder Korrektheit.
    \end{center}
    \thispagestyle{empty}
    \newpage

    \tableofcontents
    ~\\
    Alle mit [*] markierten Kapitel sind noch nicht Korrektur gelesen und bedürfen eventuell noch Änderungen.

    \newpage


    \section{[*] Das eindimensionale Riemann-Integral}

    \subsection{Das Riemann-Integral}
    \thispagestyle{pagenumberonly}
    \marginnote{[16. Apr]}
    Frage: Was ist die Fläche unter einem Graphen?
    % TODO: Vis
    % TODO: Konstant.
    \begin{definition}[Zerlegung]
        Eine Zerlegung $Z$ eines kompakten Intervalls $I=\interv{a,b}$ in Teilintervalle $I_j$ ($j=1,\dots, k$) der Längen $\abs{I_j}$ ist eine Menge von Punkten $x_0$, $x_1$,$\dots$, $x_k\in I$ (Teilpunkte von $Z$) mit
        \begin{align*}
            a=x_0 < x_1 < x_2 < \dots < x_k = b
        \end{align*}
        und $I_j = \interv{x_{j-1}, x_j}$. Wir setzen $\varDelta x_j \definedas x_j - x_{j-1} \definedasbackwards\abs{I_j}$.
    \end{definition}
    \begin{definition}[Feinheit einer Zerlegung]
        Die Feinheit der Zerlegung $Z$ ist definiert als die Länge des längsten Teilintervalls von $Z$:
        \begin{align*}
            \varDelta\of{z}\definedas \max\of{\abs{I_1}, \abs{I_2}, \dots, \abs{I_k}} = \max\of{\varDelta x_1, \varDelta x_2, \dots, \varDelta x_k}
        \end{align*}
    \end{definition}
    \begin{notation}[Riemannsche Zwischensumme]
        Wir setzen
        \begin{align*}
            B\of{I} = \set{f: I\fromto \R~\middle\vert~\sup_{x\in I} \abs{f(x)} < \infty}
        \end{align*}
        als die Menge aller beschränkten reellwertigen Funktionen auf $I$. In jedem $I_j$ wählen wir ein $\xi_j\in I_j$ und setzen $\xi=\pair{\xi_1, \xi_2, \dots, \xi_k}$. Für $f\in B\of{I}$ setzen wir die Riemannsche Zwischensumme
        \begin{align*}
            S_Z\of{f} &= S_Z\of{f, \xi} \definedas \sum_{j=1}^{k} f\of{\xi_j}\cdot\varDelta x_j = \sum_{j=1}^{k} f\of{\xi_j}\cdot\abs{I_j}
        \end{align*}
    \end{notation}
    \begin{notation}[Ober- und Untersumme]
        Für $f\in B\of{I}$ setzen wir
        \begin{align*}
            \underline{m}_j &\definedas \inf_{I_j} f = \inf\set{f(x): x\in I_j}\\
            \overline{m}_j &\definedas \sup_{I_j} f = \sup\set{f(x): x\in I_j}\\
            \overline{S}_Z\of{f}&\definedas \sum_{j=1}^{k} \overline{m}_j\cdot \varDelta x_j\tag{Obersumme}\\
            \underline{S}_Z\of{f}&\definedas \sum_{j=1}^{k} \underline{m}_j\cdot \varDelta x_j\tag{Untersumme}
            \intertext{Damit gilt für $x \in I_j$}
            \underline{m}_j &\leq f\of{x} \leq \overline{m}_j\\
            \impl \underline{m}_j &\leq f\of{\xi_j} \leq \overline{m}_j\\
            \impl \underline{S}_Z\of{f} &\leq S_Z\of{f, \xi} \leq \overline{S}_Z\of{f}
        \end{align*}
    \end{notation}
    \horizontalline \noindent Wir wollen die Zerlegung $Z$ systematisch verfeinern.
    \begin{definition}[Verfeinerung einer Zerlegung]
        Eine Zerlegung $Z^{*}$ von $I$ ist eine Verfeinerung der Zerlegung $Z$ von $I$, falls alle Teilpunkte von $Z$ auch Teilpunkte von $Z^{*}$ sind.
    \end{definition}
    % Vis
    \begin{definition}[Gemeinsame Verfeinerung]
        Die gemeinsame Verfeinerung $Z_1 \lor Z_2$ zweier Zerlegungen $Z_1, Z_2$ von $I$ ist die Zerlegung von $I$, deren Teilpunkte gerade die Teilpunkte von $Z_1$ und $Z_2$ sind.
    \end{definition}
    % Vis

    \begin{lemma} % Lemma 3
        \label{lemma:verfeinerung-ober-unter-summe}
        Ist $Z^{*}$ eine Verfeinerung der Zerlegung $Z$ von $I$ und $f\in B\of{I}$. Dann gilt
        \begin{align*}
            \underline{S}_Z\of{f} \leq \underline{S}_{Z^{*}}\of{f} \leq \overline{S}_{Z^{*}}\of{f} \leq \overline{S}_Z\of{f}
        \end{align*}
        \begin{proof}
            $Z^{*}$ enthält alle Teilpunkte von $Z$, nur mehr.\\
            \textsc{Schritt 1:} Angenommen $Z^{*}$ enthält genau einen Teilpunkt ($y_{\pair{l+1}}$) mehr als $Z$. Das heißt
            \begin{align*}
                y_j &= x_j\quad\fa 0\leq j \leq l\\
                x_l < y_{l+1} &< x_{l+1}\\
                y_{j+1} &= x_j\quad \fa l+ 1 \leq j \leq k
                \intertext{Dann gilt}
                \underline{S}_Z\of{f} &= \sum_{j=1}^{k} \underline{m}_j\varDelta x_j = \sum_{j=1}^{l} \underline{m}_j \varDelta x_j + \underline{m}_{l+1}\varDelta x_{l+1} + \sum_{j=l+2}^{k} \underline{m}_j \varDelta x_j\\
                \underline{m}_j &= \inf_{I_j} f = \inf_{I_j^{*}} f = \underline{m}_j^{*}\quad \fa 0 \leq j \leq l\\
                \underline{m}_j &= \inf_{I_j} f = \inf_{I_{j+1}^{*}}f = \underline{m}_{j+1}^{*}\quad \fa j\geq l+2\\
                I_j &= \interv{x_j, x_{j-1}} = \interv{y_{j+1}, y_j} = I_{j+1}^{*}\quad\fa j \geq l+2\\
                \impl \sum_{j=l+2}^{k} \underline{m}_j \varDelta x_j &= \sum_{j=l+2}^{k} \underline{m}_{j+1}^{*} \varDelta y_{j+1} = \sum_{j=l+3}^{k+1} \underline{m}_j^{*} \varDelta y_{j}\\
                \underline{m}_{l+1} \varDelta x_{l+1} &= \underline{m}_{l+1}\of{x_{l+1} - x_l} = \underline{m}_{l+1}\of{y_{l+2}-y_l}\\
                &= \underline{m}_{l+1}\of{y_{l+2} - y_{l+1} + y_{l+1} - y_l}\\
                &= \underline{m}_{l+1} \varDelta y_{l+2} + \underline{m}_{l+1}\varDelta y_{l+1}\\
                &\leq \underline{m}_{l+2}^{*}\varDelta y_{l+2} + \underline{m}_{l+1}^{*}\varDelta y_{l+1}
                \intertext{Insgesamt ergibt sich}
                \underline{S}_{Z}\of{f} &\leq \sum_{j=1}^{l} \underline{m}_j^{*} \varDelta y_j + \underline{m}_{l+1}^{*} \varDelta y_{l+1} + \underline{m}_{l+2}^{*}\varDelta y_{l+2} + \sum_{j=l+3}^{k+1} \underline{m}_j^{*}\varDelta y_j = \underline{S}_{Z^{*}}\of{f}
            \end{align*}
            ähnlich zeigt man $\overline{S}_Z\of{f} \geq \overline{S}_{Z^{*}}\of{f}$.\\[10pt]
            \textsc{Schritt 2:} Sei $Z^{*}$ eine beliebige Verfeinerung von $Z$. Wir nehmen eine endliche Folge von Einpunkt-Verfeinerungen $Z=Z_0, Z_1, Z_2, \dots, Z_r = Z^{*}$. Dabei hat $Z_{s+1}$ genau einen Punkt mehr als $Z_{s}$. Dann gilt nach \textsc{Schritt 1}, dass
            \begin{align*}
                \underline{S}_{Z}\of{f} &\leq \underline{S}_{Z_1}\of{f} \leq \dots \leq \underline{S}_{Z^{*}}\of{f}\\
                \overline{S}_{Z}\of{f} &\geq \overline{S}_{Z_1}\of{f} \geq \dots \geq \overline{S}_{Z^{*}}\of{f}
            \end{align*}
            \textsc{Schritt 3:} Sei $\xi^{*} = \pair{\xi_1^{*}, \xi_2^{*}, \dots, \xi_l^{*}}$ der Zwischenpunkt zur Zerlegung $Z^{*}$. Dann gilt
            \begin{align*}
                S_{Z^{*}}\of{f} \leq S_{Z^*}\of{f, \xi^{*}} &\leq \overline{S}_{Z^*}\of{f}\qedhere
            \end{align*}
        \end{proof}
    \end{lemma}

    \begin{lemma} % Lemma 4
        \label{lemma:ober-unter-summe-zerlegung}
        Seien $Z_1$, $Z_2$ Zerlegungen von $I$. Dann gilt
        \begin{align*}
            \underline{S}_{Z_1}\of{f} \leq \overline{S}_{Z_2}\of{f}\qquad\forall f\in B\of{I}
        \end{align*}
        \begin{proof}
            Es gilt nach Lemma~\ref{lemma:verfeinerung-ober-unter-summe}, dass
            \begin{align*}
                \underline{S}_{Z_1}\of{f} &\leq \underline{S}_{Z_1\lor Z_2}\of{f}\leq \overline{S}_{Z_1\lor Z_2}\of{f} \leq \overline{S}_{Z_2}\of{f}\qedhere
            \end{align*}
        \end{proof}
    \end{lemma}

    \begin{bemerkung}
        Für $I=\interv{a,b}$ und $f\in B\of{I}$ gilt immer
        \begin{align*}
            \abs{I}\cdot \inf_{I} f \leq \underline{S}_Z\of{f} \leq \overline{S}_Z\of{f} \leq \abs{I}\cdot \sup_{I} f
        \end{align*}
        für alle Zerlegungen $Z$ von $I$. Somit sind
        \begin{align*}
            \set{\overline{S}_Z\of{f} : Z \text{ ist eine Zerlegung von } I}
            \intertext{und}
            \set{\underline{S}_Z\of{f} : Z \text{ ist eine Zerlegung von } I}
        \end{align*}
        beschränkte, nicht-leere Teilmengen von $\R$.
    \end{bemerkung}

    \begin{definition}[Ober- und Unterintegral]
        Es sei $I=\interv{a,b}$ und $f\in B\of{I}$. Dann definieren wir
        \begin{align*}
            \overline{J}\of{f} \definedas \inf\set{\overline{S}_Z\of{f} : Z\text{ ist Zerlegung von $I$ } }\tag{Oberintegral}\\
            \underline{J}\of{f} \definedas \sup\set{\underline{S}_Z\of{f} : Z\text{ ist Zerlegung von $I$ } }\tag{Unterintegral}
        \end{align*}
    \end{definition}

    \begin{lemma} % Lemma 6
        \label{lemma:temp-6}
        Es sei $Z$ eine Zerlegung von $I$. Dann gilt
        \begin{align*}
            \underline{S}_{Z}\of{f} \leq \underline{J}\of{f} \leq \overline{J}\of{f} \leq \overline{S}_{Z}\of{f}
        \end{align*}
        \begin{proof}
            Nach Lemma~\ref{lemma:ober-unter-summe-zerlegung} gilt für zwei beliebige Zerlegungen $Z_1$, $Z_2$
            \begin{align*}
                \underline{S}_{Z_1}\of{f} &\leq \overline{S}_{Z_2}\of{f}
                \intertext{Wir fixieren $Z_2$ und erhalten}
                \impl \sup\set{\underline{S}_{Z_1}\of{f}: Z_1 \text{ ist eine Zerlegung von } I} &\leq \overline{S}_{Z_2}\of{f}\\
                \impl \underline{J}\of{f} &\leq \overline{S}_{Z_2}\of{f}\\
                \impl \underline{J}\of{f} &\leq \inf\set{\overline{S}_{Z_2}\of{f}: Z_2 \text{ ist eine Zerlegung von } I}\\
                \impl \underline{J}\of{f} &\leq \overline{J}\of{f}\\
                \impl\underline{S}_{Z}\of{f} \leq \underline{J}\of{f}&\leq \overline{J}\of{f} \leq \overline{S}_Z\of{f}\qedhere
            \end{align*}
        \end{proof}
    \end{lemma}

    \begin{definition}[Integral]
        Es sei $I=\interv{a,b}$. $f\in B\of{I}$ heißt (Riemann-)integrierbar, falls
        \begin{align*}
            \underline{J}\of{f} = \overline{J}\of{f}
        \end{align*}
        In diese Fall nennen wir $J(f) \definedas \underline{J}\of{f} = \overline{J}\of{f}$ das bestimmte Integral von $f$ über $\interv{a,b}$ und schreiben
        \begin{align*}
            \int_{a}^{b} f(x) \dif x = \int_{a}^{b} f\dif x = \int_{I} f(x)\dif x = \int_{I} f\dif x \definedasbackwards J(f)
        \end{align*}
        Die Klasse der Riemann-integrierbaren Funktionen $f\in B\of{I}$ nennen wir $R(I)$.
    \end{definition}

    \begin{beispiel}[Konstante Funktion]
        \marginnote{[18. Apr]}
        $f(x) \definedas c$ auf $\interv{a,b}$ für eine Konstante $c\in\R$. Dann gilt
        \begin{align*}
            \int_{a}^{b} f\of{x} \dif x &= c\cdot\pair{b-a}
        \end{align*}
    \end{beispiel}

    \begin{beispiel}
        Die Funktion $f: \interv{0,1}\fromto\R$
        \begin{align*}
            f(x) \definedas \begin{cases}
                                1 &x\in \Q\\
                                0 &\text{sonst}
            \end{cases}
        \end{align*}
        ist nicht Riemann-integrierbar, weil $\overline{J}\of{f} = 1$ und $\underline{J}\of{f} = 0$.
    \end{beispiel}

    \subsection{Integrabilitätskriterien}

    \begin{satz}[1. Kriterium]
        \label{satz:integr-kriterium-1}
        Es sei $f\in B\of{I}$. Dann gilt $f\in R\of{I}$ genau dann, wenn
        \begin{align*}
            \fa\varepsilon > 0\ex\text{Zerlegung } Z\text{ von } I\text{ mit } \overline{S}_{Z}\of{f}-\underline{S}_{Z}\of{f} < \varepsilon
        \end{align*}
        \begin{proof}
            \anf{$\Leftarrow$} Nach Lemma~\ref{lemma:temp-6} gilt
            \begin{align*}
                \underline{S}_Z\of{f}\leq \underline{J}\of{f} &\leq \overline{J}\of{f} \leq \overline{S}_Z\of{f}
                \intertext{Sei $\varepsilon > 0$, dann gilt}
                0&\leq\overline{J}\of{f}-\underline{J}\of{f}\leq\overline{S}_Z\of{f} - \underline{S}_Z\of{f} < \varepsilon\\
                \impl 0&\leq \overline{J}\of{f} - \underline{J}\of{f} \leq 0\\
                \impl f&\in R\of{I}
            \end{align*}
            \anf{$\impl$} Angenommen $f\in R\of{I}$, das heißt
            \begin{align*}
                \overline{J}\of{f} &= \underline{J}\of{f}\\
                \overline{J}\of{f} &= \inf\set{\overline{S}_Z\of{f}: Z \text{ ist eine Zerlegung von } I}\\
                \underline{J}\of{f} &=\sup\set{\underline{S}_Z\of{f}: Z \text{ ist eine Zerlegung von } I}
                \intertext{Das heißt zu $\varepsilon > 0$ existieren Zerlegungen $Z_1$, $Z_2$ von $I$ mit}
                \overline{J}\of{f} + \frac{\varepsilon}{2} &> \overline{S}_{Z_1}\of{f}\\
                \underline{J}\of{f} - \frac{\varepsilon}{2} &< S_{Z_2}\of{f}
                \intertext{Da $f\in R\of{I}$ gilt $\underline{J}\of{f} = \overline{J}\of{f}$. Wir definieren die gemeinsame Verfeinerung $Z\definedas Z_1 \lor Z_2$. Dann gilt}
                \overline{S}_Z\of{f} - \underline{S}_Z\of{f} &< \overline{J}\of{f} + \frac{\varepsilon}{2} - \pair{\underline{J}\of{f} - \frac{\varepsilon}{2}}\\
                &= \overline{J}\of{f} - \underline{J}\of{f} + \frac{\varepsilon}{2} + \frac{\varepsilon}{2} = \varepsilon\qedhere
            \end{align*}
        \end{proof}
    \end{satz}

    \begin{satz}[2. Kriterium]
        Sei $f\in B\of{I}$. Dann gilt $f\in R\of{I}$ genau dann, wenn
        \begin{align*}
            \fa\varepsilon > 0\ex\delta > 0\fa \text{Zerlegungen }Z \text{ von } I \text{ mit Feinheit } \Delta\of{Z} < \delta\colon \overline{S}_Z\of{f} - \underline{S}\of{f} < \varepsilon
        \end{align*}
        \begin{proof}
            \anf{$\Leftarrow$} wird von Satz~\ref{satz:integr-kriterium-1} bereits impliziert.\\[10pt]
            \anf{$\impl$} Sei $f\in R\of{I}$ und $\varepsilon > 0$. Dann gilt nach Satz~\ref{satz:integr-kriterium-1}, dass eine Zerlegung $Z'=\pair{x_0', x_1', \dots, x_l' = b}$ von $I$ mit
            \begin{align*}
                \overline{S}_Z\of{f} - \underline{S}_Z\of{f} &< \frac{\varepsilon}{2}
                \intertext{existiert. Wähle eine andere Zerlegung $Z$ von $I$ mit $\Delta\of{Z} < \delta$, wobei $\delta > 0$ noch später gewählt wird. Setze $Z^{*} = Z'\lor Z$. Nach Lemma~\ref{lemma:verfeinerung-ober-unter-summe} und Satz~\ref{satz:integr-kriterium-1} gilt}
                \overline{S}_{Z^{*}}\of{f} - \underline{S}_{Z^{*}}\of{f} &< \frac{\varepsilon}{2}
                \intertext{Wir wollen die Ober- und Untersumme von $Z^{*}$ mit denen in $Z$ vergleichen.}
                \overline{S}_Z\of{f} - \underline{S}_{Z^{*}}\of{f} &= \sum_{j} \overline{m}_j\cdot\abs{I_j} - \sum_{t}^{} \overline{m}_t\cdot\abs{I_t}
                \intertext{wobei $I_j = \interv{x_{j-1}, x_j}$. Da $Z^*$ eine Verfeinerung von $Z$ ist, sind alle Teilpunkte von $Z$ auch Teilpunkte von $Z^*$. Das heißt die Intervalle $I_j$ (zu $Z$) unterscheiden sich von den Intervallen $I_j^*$ (zu $Z^*$) sofern Punkte $x_{\nu}'$ (Teilpunkte von $Z^*$) im Inneren von $I_j$ liegen. Also gilt}
                I_Z^{*} \cap I_j &\neq \emptyset \impl I_Z^{*} \subseteq I_j
                \intertext{Frage: Wie viele Intervalle $I_j$ existieren maximal, für die $I_j$ eine Verfeinerung von $Z$ oder ? hinter reellen $I_j^*$ ist? Dann muss mindestens ein Punkt von der Zerlegung $Z'$ unterhalb von $I_j$ liegen. Wir haben $l$ Punkte in Zerlegung $Z'$. Das heißt die Anzahl solcher Intervalle $I_j$ ist maximal $l$.}
                \overline{S}_Z\of{f} - \overline{S}_{Z^{*}}\of{f} &= \sum_{j}^{} \overline{m}_j\cdot\abs{I_j} - \sum_{t}^{} \overline{m}_t^{*} \cdot\abs{I_j^{*}}\\
                &= \sum_{j}^{} \pair{\overline{m}_j \cdot\abs{I_j} - \sum_{t: I_{Z}^{*} \subseteq I_j}^{} \overline{m}_t^{*} \cdot\abs{I_t^{*}}}\\
                &= \sum_{j}^{} \sum_{t: I_t^{*}}^{} \pair{\overline{m}_j - \overline{m}_t^{*}}\cdot\abs{I_t^{*}}\\
                \overline{S}_Z\of{f} - \overline{S}_Z\of{f} &= \sum_{j}^{} \sum_{t: I_t^{*}}^{} \pair{\underbrace{\overline{m}_j - \overline{m}_t^{*}}_{= 0 \text{ falls } I_t^{*} = I_j}}\cdot\abs{I_t^{*}}\\
                &= \sum_{j}^{} \sum_{t: I_t^{*}}^{} \pair{\overline{m}_j - \overline{m}_t^{*}}\cdot\abs{I_Z^{*}}\\
                f(x) &= f(y) + f(x) - f(y)\\
                &\leq f(y) + \sup_{s_1, s_2\in I}\set{f(s_1) - f(s_2)}\\
                f(x) &\leq f(y) + 2\norm{f}_{\infty}
                \intertext{genauso}
                f(x) &= f(y) + f(x) - f(y)\\
                &\geq f(y) + \inf_{s_1, s_2\in I}\set{f(s_1) - f(s_2)}\\
                &\geq f(y) - 2\norm{f}_{\infty}\\[10pt]
                \impl \overline{m}_j &= \sup_{s\in I_j}f(x) \leq 2\norm{f}_{\infty} + f(y)\quad\forall y\in I_t^{*}\\
                \impl \overline{m}_j &\leq 2\norm{f}_{\infty} + \sup_{?} f = 2\norm{f}_{\infty} + \overline{m}_z^{*}\\
                \vdots \quad &???
                \intertext{Genauso zeigt man}
                \underline{S}_Z\of{f} - \underline{S}_{Z^{*}}\of{f} &\geq -2\norm{f}_{\infty} l\cdot\delta\\
                \impl \overline{S}_Z\of{f} &\leq \overline{S}_{Z^{*}} + 2\norm{f}_{\infty} l \cdot \delta\\
                \underline{S}_Z\of{f} &\geq \underline{S}_{Z^{*}} - 2 \norm{f}_{\infty} l \cdot \delta\\
                \impl \overline{S}_Z\of{f} - \underline{S}_Z\of{f} &\leq \overline{S}_{Z^{*}}\of{f} + 2\norm{f}_{\infty} l\delta - \pair{\underline{S}_{Z^{*}}\of{f} - 2 \norm{f}_{\infty} l \cdot\delta}\\
                &= ?\\
                &< \frac{\varepsilon}{2} + 4\norm{f}_{\infty} l \cdot\delta
                \intertext{Jetzt wähle $\delta = \frac{\varepsilon}{\delta\pair{\norm{f}_{\infty} + 1}\cdot l}$}
                \impl &\leq \frac{\varepsilon}{2} + 4\norm{f}_{\infty} \cdot \frac{\varepsilon}{\delta\pair{\norm{f}+1}\cdot l} < \frac{\varepsilon}{2} + \frac{\varepsilon}{2} = \varepsilon
            \end{align*}
            sofern um $\Delta\of{z} < \delta$ ist.
        \end{proof}
    \end{satz}

    \begin{anwendung}
        Zerlegung $Z_n$ von $I$ mit Feinheit $\Delta\of{Z_n}\fromto 0$ für $\ntoinf$. $\xi_n$ Zwischenpunkt von Zerlegung $Z_n$. Die Riemannnsumme
        \begin{align*}
            S_{z_n}\of{f, \xi_n} &= \sum_{j=1}^{k_n} f\of{\xi_j^n}\cdot\abs{I_j^n}
        \end{align*}
        konvergiert gegen $J(f)$ falls $f\in R\of{I}$.
    \end{anwendung}

    \begin{bemerkung}
        \marginnote{[19. Apr]}
        Sei $z=\pair{x_0, x_1, \dots, x_k}$ Zerlegung von $I=\interv{a,b}$ und $\zeta = \pair{\zeta_1, \zeta_2, \dots, \zeta_k}$ Zwischenpunkt zur Zerlegung $Z$, sodass
        \begin{align*}
            x_{j-1}&\leq \zeta_j \leq x_j\quad\fa j=1,\dots, k
            \intertext{Dann ist die Riemannsche Zwischensumme}
            S_Z\of{f} &= \sum_{j=1}^{k} f\of{\zeta_j}\cdot\abs{I_j}
        \end{align*}
        linear in $f$.
    \end{bemerkung}

    \begin{korollar} % Korollar 10
        \label{korollar:temp-10}
        Sei $f\in B\of{I}$. Dann gilt $f\in R\of{I}$ genau dann, wenn für jede Folge $(Z_n)_n$ von Zerlegungen $Z_n$ von $I$ mit Feinheit $\Delta\of{z_n}\fromto 0$ für $\ntoinf$ und jede Folge $(\xi_n)_n$ von Zwischenpunkten $\xi_n$ zugehörig zu $Z_n$ ein Grenzwert $\biglim{\ntoinf} S_{Z_n}\of{f, \xi_n}$ existiert.\\
        Darüber hinaus ist in diesem Fall obiger Grenzwert unabhängig von der Wahl der Zerlegung $Z_n$ und der Zwischenpunkten $\xi_n$ und es gilt
        \begin{align*}
            \int_{a}^{b} f\of{x} \dif x &= \lim_{\ntoinf} S_{Z_n}\of{f, \xi_n}
        \end{align*}
        \begin{proof}
            \anf{$\impl$} Sei $f\in R\of{I}$ zu $\varepsilon > 0\ex\delta > 0\colon \overline{S}_Z\of{f} - \underline{S}_Z\of{f} < \varepsilon\fa\text{Zerlegungen } Z \text{ von } I \text{ mit } \Delta\of{z} < \delta$. Da $\Delta\of{z_n}\fromto 0$ für $\ntoinf$
            \begin{align*}
                \impl \ex N\in\N_0\colon \Delta\of{Z_n} &< \delta\quad\forall n\geq N
                \intertext{und}
                \underline{S}_Z\of{f} \leq \underline{J}\of{f} &= J\of{f} \leq \overline{S}_Z\of{f}\\
                \underline{S}_Z\of{f} \leq S_{Z_n}\of{f,\xi_n} &\leq \overline{S}_{Z_n}\of{f}\\
                \impl \abs{J\of{f} - S_{Z_n}\of{f, \xi_n}} &< \varepsilon\quad\forall n\geq N
                \intertext{das heißt}
                \lim_{\ntoinf} S_{Z_n}\of{f, \xi_n} &= J\of{f} = \int_{a}^{b} f \dif x
            \end{align*}
            \anf{$\Leftarrow$} \textsc{Schritt 1:} Angenommen $\biglim{\ntoinf} S_{Z_n}\of{f, \xi_n}$ existiert für jede Folge $\pair{Z_n}_n$ von Zerlegungen von $I$ mit $\Delta\of{Z_n}\fromto 0$ und jede Wahl von Zwischenpunkten $\pair{\xi_n}_n$ zu $Z_n$.\\
            Seien $\pair{Z_n^{1}}_n$, $\pair{Z_n^{2}}_n$ zwei solche Folgen von Zerlegungen mit $\pair{\xi_n^1}_n$, $\pair{\xi_n^2}_n$ zugehörigen Folgen von Zwischenpunkten. Sei $\pair{Z_n}_n$ eine neue Folge von Zerlegungen von $I$, wobei $Z_{2k} = Z_k^2$ und $Z_{2k-1} = Z^1_k$, außerdem sei $\xi_{2k} = \xi^2_k$ und $\xi_{2k-1}=\xi^1_k$. Dann wissen wir, dass
            \begin{align*}
                \lim_{\ntoinf} &S_{Z_n}\of{f, \xi_n}
                \intertext{existiert und gilt}
                \lim_{\ntoinf} S_{Z_n}\of{f, \xi_n} &= \lim_{\ntoinf} S_{Z_{2n}}\of{f, \xi_{2n}}\\
                &= \lim_{\ntoinf} S_{Z_{2n-1}}\of{f, \xi_{2n-1}}\\
                &= \lim_{\ntoinf} S_{Z_n^2}\of{f, \xi_n^2}\\
                &= \lim_{\ntoinf} S_{Z_n^1}\of{f, \xi_n^1}
            \end{align*}
            \textsc{Schritt 2:} (Später)
        \end{proof}
    \end{korollar}
    
    \begin{satz} % Satz 11
        Der Raum $R\of{I}$ auf einem kompakten Intervall $I=\interv{a,b}$ ist ein Vektorraum und $J: R\of{I}\fromto\R~~f\mapsto J\of{f} = \int_{a}^{b} f \dif x$ ist eine lineare Abbildung.\\
        Für $f,g\in R\of{I}$ und $\alpha,\beta\in\R$ folgt $\alpha f + \beta g \in R\of{I}$ und $J\of{\alpha f + \beta g} = \alpha J\of{f} + \beta J\of{g}$.
        \begin{proof}
            \textsc{Schritt 1:} Sei $h: I\fromto\R$ und Zerlegung $Z$ von $I$ mit zugehörigen Intervallen $Ij$. Dann gilt
            \begin{align*}
                \overline{m}_j = \sup_{x\in I_j}h(x)&\quad \underline{m}_j = \inf_{y\in I_j} h(y)\\
                \impl \overline{m}_j - \underline{m}_j &= \sup_{x\in I_j}h(x) - \inf_{y\in I_j} h(y)\\
                &= \sup_{x\in I_j} h(x) + \sup_{y\in I_j}\pair{-h\of{y}}\\
                &= \sup_{x,y\in I_j}\pair{h\of{x}-h\of{y}}\\
                &= \sup_{x,y\in I_j}\pair{h(y) - h(x)}\tag{Vertauschen von $x,y$}\\
                &= \sup_{x,y\in I_j}\pair{\abs{h(x)-h(y)}}\\
                \overline{m}_j\of{h} - \underline{m}_j\of{h} &= \sup_{x,y\in I_j}\pair{\abs{h(x)-h(y)}}\tag{1}
                \intertext{Nehmen $h=\alpha f + \beta g$; $f,g\in R\of{I}$; $\alpha, \beta\in\R$}
                h(x) - h(y) &= \alpha\pair{f(x)-f(y)} + \beta\pair{g(x)-g(y)}\\
                \abs{h(x)-h(y)} &\leq \abs{\alpha}\abs{f(x)-f(y)}+ \abs{\beta}\abs{g(x)-g(y)}\\
                \impl \overline{m}_j\of{h} - \underline{m}_j\of{h} &= \sup_{x\in I_j} h(x) - \inf_{y\in I_j} h(y)\\
                \annot[{&}]{=}{(1)} \sup_{x,y\in I_j} \pair{\abs{h(x)-h(y)}}\\
                &\leq \abs{\alpha}\cdot \sup_{x,y\in I_j}\abs{f(x)-f(y)} + \abs{\beta}\cdot\sup_{x,y\in I_j} \abs{g(x)-g(y)}\\
                \impl \overline{S}_Z\of{h} - \underline{S}_{Z}\of{h} &= \sum_{j}^{} \pair{\overline{m}_j\of{h} - \underline{m}_j\of{h}]}\abs{I_j}\\
                &\leq \abs{\alpha}\sum_{j}^{} \pair{\overline{m}_j\of{f} - \underline{m}_j\of{f}}\abs{I_j} + \abs{\beta} \sum_{j}^{} \pair{\overline{m}_j\of{g} - \underline{m}_j\of{g}}\abs{I_j}\\
                \impl \overline{S}_Z\of{h} - \underline{S}_Z\of{h} &\leq \abs{\alpha}\pair{\overline{S}_Z\of{f} - \underline{S}_Z\of{f}} + \abs{\beta}\pair{\overline{S}_Z\of{g} - \underline{S}_Z\of{g}}\\
                \impl \fa\varepsilon > 0\ex Z_1\colon \overline{S}_{Z_1}\of{f} - \underline{S}_{Z_1}\of{f} &< \frac{\varepsilon}{2\pair{1+\abs{\alpha} + \abs{\beta}}}\\
                \fa\varepsilon > 0\ex Z_2\colon \overline{S}_{Z_2}\of{g} - \underline{S}_{Z_2}\of{g} &< \frac{\varepsilon}{2\pair{1+\abs{\alpha} + \abs{\beta}}}\\
                \intertext{Wähle $Z=Z_1\lor Z_2$}
                \impl \overline{S}_Z\of{h} - \underline{S}_Z\of{h} &< \abs{\alpha}\frac{\varepsilon}{2\pair{1+\abs{\alpha}+\abs{\beta}}} + \abs{\beta}\frac{\varepsilon}{2\pair{1+\abs{\alpha}+\abs{\beta}}}\\
                &\leq \frac{\varepsilon}{2} +\frac{\varepsilon}{2} = \varepsilon
            \end{align*}
            nach Satz~\ref{satz:integr-kriterium-1} ist $h=\alpha f + \beta g$ Riemann-integrierbar.\\
            \textsc{Schritt 2:} Für Zwischensummen
            \begin{align*}
                S_Z\of{h, \xi} &= \sum_{j}^{} h\of{\xi_j}\abs{I_j}\\
                &= \alpha S_Z\of{f, \xi} + \beta S_Z\of{g, \xi}
            \end{align*}
            haben wir Linearität!\\
            Für $h,f,g\in R\of{I}$ gilt nach Korollar~\ref{korollar:temp-10}
            \begin{align*}
                J\of{h} &= \lim_{n\toinf} S_{Z_n}\of{h, \xi_n}\\
                &= \lim_{n\toinf}\pair{\alpha S_{Z_n}\of{f, \xi_n} + \beta S_{Z_n}\of{g, \xi_n}}\tag{$\Delta\of{Z_n}\fromto 0$}\\
                \annot[{&}]{=}{\ref{korollar:temp-10}}] \alpha \lim S_{Z_n}\of{f, \xi_n} + \beta \lim_{\ntoinf} S_{Z_n}\of{g, \xi_n}\\
                &= \alpha J\of{f} + \beta J\of{g}\qedhere
            \end{align*}
        \end{proof}
    \end{satz}

    \begin{satz} % Satz 12
        Seien $f, g\in R\of{I}$. Dann folgt $f\cdot g\in R\of{I}$ sowie $\abs{f} \in R\of{I}$. Ist außerdem $\abs{g} \geq c > 0$ auf $I$ für ein konstantes $c>0$, so ist auch $\frac{f}{g}\in R\of{I}$.
        \begin{proof}
            Es sei $h(x) = f(x)\cdot g(x)$ für $x\in I$. Dann gilt
            \begin{align*}
                \abs{h(x) - h(y)} &= \abs{f(x)g(x) - f(y)g(y)}\\
                &= \abs{g(x)\pair{f(x)-f(y)} + f(y)\pair{g(x)-g(y)}}\\
                &\leq \norm{g}_{\infty}\cdot\abs{f(x)-f(y)} + \norm{f}_{\infty} \cdot \abs{g(x)-g(y)}\\
                \norm{f}_{\infty} &= \sup_{x\in I}\abs{f(x)} < \infty
                \intertext{$Z$ Zerlegung von $I$ ist $I_j$; Teilintervalle}
                \overline{S}_Z\of{h} - \underline{S}_Z\of{h} &= \sum_{j}^{} \pair{\overline{m}_j\of{h} - \underline{m}_j\of{h}}\cdot\abs{I_j}\\
                \overline{m}_j\of{h} - \underline{m}_j\of{h} &= \sup_{I_j} h - \inf_{I_j} h = \sup_{x,y\in I_j} \abs{h(x)-h(y)}\\
                &\leq \norm{g}_{\infty} \pair{\overline{m}_j\of{f}- \underline{m}_j\of{f}} + \norm{f}_{\infty} \pair{\overline{m}_j\of{y} - \underline{m}_j\of{g}}\\
                \overline{S}_Z\of{h} - \underline{S}_Z\of{h} &\leq \norm{f}_{\infty}\cdot\abs{\overline{S}_Z\of{g} - \underline{S}_Z\of{g}} + \norm{g}_{\infty}\pair{\overline{S}_Z\of{f} - \underline{S}_Z\of{f}}
                \intertext{Zu $\varepsilon > 0$}
                \exists Z_1\colon \overline{S}_{Z_1}\of{f} - \underline{S}_{Z_1}\of{f} &< \frac{\varepsilon}{2\pair{1+\norm{f}_{\infty}}}\\
                \exists Z_2\colon \overline{S}_{Z_2}\of{f} - \underline{S}_{Z_2}\of{f} &< \frac{\varepsilon}{2\pair{1+\norm{g}_{\infty}}}
                \intertext{Es sei $Z\:= Z_1 \lor Z_2$}
                \impl \overline{S}_Z\of{h} - \underline{S}_Z\of{h} &\leq \norm{f}_{\infty}\cdot\pair{\overline{S}_Z\of{g} - \underline{S}_Z\of{g}} + \norm{g}_{\infty} \cdot\pair{\overline{S}_Z\of{f} - \underline{S}_Z\of{f}} \leq \frac{\varepsilon}{2} + \frac{\varepsilon}{2} = \varepsilon
            \end{align*}
            Das heißt $h=f\cdot g\in R\of{I}$ nach Satz~\ref{satz:integr-kriterium-1}.\\
            Für $\abs{f}$ verwende
            \begin{align*}
                \abs{\abs{f(x)}- \abs{f(y)}} &\leq \abs{f(x) - f(y)}\\
                \impl \overline{m}_j\of{\abs{f}} - \underline{m}_j\of{\abs{f}} &= \sup_{x,y\in I_j}\pair{\abs{\abs{f(x)} - \abs{f(y)}}}\\
                &\leq \sup_{x,y\in I_j} \pair{\abs{f(x)-f(y)}}\\
                &= \overline{m}_j\of{f} - \underline{m}_j\of{f}
            \end{align*}
            wie vorher folgt also $\abs{f} \in R\of{I}$.\\
            Für $\frac{f}{g}$ muss nur $\frac{1}{g}$ betrachtet und die Multiplikationsregel angewendet werden. Es gilt
            \begin{align*}
                \abs{\frac{1}{g(x)}- \frac{1}{g(y)}} &= \frac{\abs{g(x)-g(y)}}{\abs{g(x)}\abs{g(y)}}\\
                &\leq \frac{1}{\varepsilon^2}\abs{g(x) - g(y)}\\
                \impl \overline{m}_j\of{\frac{1}{y}}  - \underline{m}_j\of{\frac{1}{y}} \leq \frac{1}{\varepsilon^2} \pair{\overline{m}_j\of{y} - \underline{m}_j\of{y}}
            \end{align*}
            Dann wie vorher.\qedhere
        \end{proof}
    \end{satz}

\end{document}
